\documentclass[]{article}
\usepackage[utf8]{inputenc}
\usepackage[brazil]{babel}
\usepackage[lmargin=3cm,tmargin=3cm,rmargin=2cm,bmargin=2cm]{geometry}
\usepackage[onehalfspacing]{setspace}

\usepackage{graphicx,xcolor,comment,enumerate,multirow,multicol,indentfirst}
\everymath{\displaystyle}
\usepackage{amsmath,amsthm,amssymb,amsfonts,dsfont,mathtools,blindtext}


\title{Numbers complex - The necessary}
\author{Vassily Isenbaev}
\date{\today}

\begin{document}

\maketitle
\begin{large}
\section{Conceitos iniciais}
\begin{flushleft}

Potências de i \vspace{.3cm}

$i^0 = 1$

$i^1 = i$

$i^2 = -1$

$i^3 = -i$

$i^4 = 1$

$i^5 = i$

$i^6 = -1$

$i^7 = -i$

$i^8 = 1$ \vspace{.3cm}

A cada 4 números contados, eles começam a se repetir.

\subsection{Formas algébricas}

Forma algébrica: $z=a+bi$.

A letra a representa a parte real, já a letra b multiplicando o i representa a parte imaginária.

Exemplos: \vspace{.3cm}

A) $2+3i$, 2 é a parte real e 3 é a parte imaginária.

B) $1-i$, 1 é a parte real e -1 é a parte imaginária.

C) $7$, 7 é a parte real e 0 é a parte imaginária.

D) $5i$, 0 é a parte real e 5 é a parte imaginária.

\subsection{Classificações - Real, imaginário e Imaginário Puro}

Real: Quando o $b = 0$.

Imaginário: Quando o $b \not= 0$.

Puro: Quando o $b \not= 0$ e $a = 0$

\subsection{Igualdade}

Dois complexos são iguais quando as partes reais forem iguais e as partes imagináris também forem iguais. \vspace{.3cm}

Exemplo: \vspace{.3cm}

\begin{center}

Se $z = w$

então

{$a = c$ e $b = d$}

\end{center}

Exemplo: \vspace{.3cm}

$z = 3+(k-5)i$ e $w = (x^2-6)-2i$ são iguais. Calcule $k$ e $x$ \vspace{.3cm}

\begin{center}

Valor de $X$

$x^2-6=3$

$x^2=9$

$x=\pm{3}$ \vspace{.3cm}

Valor de $K$

$k-5 = -2$

$k=3$
\end{center} 

\subsection{Operações - Adição, subtração, multiplicação e divisão}

Somamos ou subtraimos a parte real com parte real e parte imaginária com parte parte imaginária. \vspace{.3cm}

Exemplo: \vspace{.3cm}

\begin{center}
    $z = 3+4i$ e $w = -1+2i$
\end{center}

Calculando $z+w$ teremos: \vspace{.3cm}

\begin{center}
    $3-1 = 2$
    
    $4i + 2i = 6i$
    
    então
    
    $z+w = 2+6i$
\end{center}

Exemplo: \vspace{.3cm}

\begin{center}
    $z = 5+i$ e $w = -1+3i$
\end{center}

Calculando $z-w$ teremos: \vspace{.3cm}

\begin{center}
    $5-(-1) = 6$
    
    $1i - 3i = 2i$
    
    então
    
    $z-w = 6+2i$
\end{center}

Exemplo: \vspace{.3cm}

\begin{center}
    $z=2+3i$ e $w=3-2i$
\end{center}

Calculando $z.w$ teremos: \vspace{.3cm}

\begin{center}
    $(2+3i).(3-2i)=$
    
    $6-4i+9i-6i^2=6+5i-6i^2$
    
\end{center}

Usando a propriedade de $i^2 = -1$ teremos: \vspace{.3cm}

\begin{center}
    $6+5i-6.-1= 12+5i$
\end{center}

Divisão: \vspace{.3cm}

Conjugado: O conjugado de $z=a+bi$ é igual a $z=a-bi$, então apenas o sinal é trocado.

Para realizar a divisão de $z por w$ faremos o seguinte: \vspace{.3cm}

\begin{center}
    $\frac{z}{w} = \frac{z}{w}.\frac{w}{w}$
\end{center}

Exemplo: \vspace{.3cm}

Seja $z=2+3i$ e $w=3-i$, calcule $z \div w$ \vspace{.3cm}

\begin{center}
    $\frac{2-3i}{3-i} \cdot \frac{3+i}{3+i} = \frac{6+2i-9i-3i^2}{3^2-i^2} = \frac{6-7i+3}{9+1} = \frac{9-7i}{10}$ \vspace{.3cm}
\end{center}

\section{Interpretação geométrica, módulo e argumento}

\includegraphics[width=12cm]{AA} \vspace{.3cm}

Exemplo: \vspace{.3cm}

\begin{center}
    $z=2+3i$ e $w=-2+2i$
\end{center}

A representação geométrica dessa equação será:

\includegraphics[width=12cm]{AAA} \vspace{.3cm}

A parte real da equação é escrita no eixo das abscissas e a parte imaginária no eixo das ordenadas.

\subsection{Modulo}

O módulo é a distancia da origem até o afixo e pode ser representade de duas maneiras: Pela letra grega $\rho$ ou $|z|$ como veremos abaixo representado na imagem.

\includegraphics[width=12cm]{AAAA} \vspace{.3cm}

A formula utilizada para achar o valor do módulo é \vspace{.3cm}
\begin{center}
    $\rho=\sqrt{a^2+b^2}$    
\end{center}


Exemplos: \vspace{.3cm}

Determine o módulo dos valores abaixo:

A) $2+3i$ \vspace{.2cm}
\begin{center}
$\rho=\sqrt{2^2+3^2}$

$\rho=\sqrt{4+9}$

$\rho=\sqrt{a^2+b^2}$  

$\rho=\sqrt{13}$ \vspace{.3cm}

\end{center}
B) $5i$ \vspace{.2cm}
\begin{center}
$\rho=\sqrt{0^2+5^2}$  

$\rho=\sqrt{25}$  

$\rho= 5$ \vspace{.3cm}
\end{center}
\subsection{Argumento}

O argumento não é nada mais do que um ângulo, ou seja, é o angulo formado entra o eixo $x$ e o eixo $y$ no sentido anti-horário como veremos na imagem a seguir:

\includegraphics[width=12cm]{AAAAA} \vspace{.3cm}

As formulas que serão usadas para encontrar o valor do argumento serão: \vspace{.3cm}

\begin{center}
    $cos\theta = \frac{a}{\rho}$
\end{center}

\begin{center}
    $sen\theta = \frac{b}{\rho}$
\end{center}

Exemplo: \vspace{.3cm}

Determine o modulo de $z=\sqrt{3}+i$
\begin{center}
    $\rho=\sqrt{\sqrt{(3)^2}+1^2}$
    
    $\rho=\sqrt{4}$
    
    $\rho=2$
    
\end{center}

Com o resultado do modulo, podemos calcular o argumento é: \vspace{.3cm}

\begin{center}
    $sen\theta=\frac{1}{2}=30^{\circ}$
    
\end{center}

\begin{center}
    $cos\theta=\frac{\sqrt{3}}{2}=30^{\circ}$
\end{center}

\begin{center}
    $\theta=30^{\circ}$ ou $\frac{\pi}{6}$
\end{center}

\includegraphics[width=12cm]{AAAAAA}

Ou seja, o angulo $\theta$ da imagem acima vale $\frac{\pi}{6}$ \vspace{.3cm}

\section{Formas trigonométricas}

Para calcular usando a forma trigonométrica usamos a formula a seguir: \vspace{.3cm}

\begin{center}
    $z=\rho(cos\theta + isen\theta)$ \vspace{.3cm}
\end{center}

Para conseguir usar essa formula, você precisa do valor de $\rho$ e do valor de $cos\theta$ e $sen\theta$. \vspace{.3cm}

Exemplo: \vspace{.3cm}

Dê a forma trigonométrica de $z=2+2i$

Primeiro temos que achar o valor do módulo: \vspace{.3cm}

\begin{center}
    $\rho=\sqrt{2^2+2^2} = \sqrt{8}$
    
    $\sqrt{8}$ pode ser escrita dessa forma $2\sqrt{2}$ \vspace{.3cm}
\end{center}

Agora vamos encontrar o valor do argumento: \vspace{.3cm}

\begin{center}
    $sen\theta=\frac{2}{2\sqrt{2}}$
\end{center}

Racionalizando, teremos: \vspace{.3cm}

\begin{center}
    $\theta=\frac{2}{2\sqrt{2}} \cdot \frac{\sqrt{2}}{\sqrt{2}} = \frac{\sqrt{2}}{2} = 45^\circ$ \vspace{.3cm}
\end{center}

Agora vamos achar o cosseno: \vspace{.3cm}

\begin{center}
    $cos\theta=\frac{2}{2\sqrt{2}}$
\end{center}

Racionalizando, teremos: \vspace{.3cm}

\begin{center}
    $\theta=\frac{2}{2\sqrt{2}} \cdot \frac{\sqrt{2}}{\sqrt{2}} = \frac{\sqrt{2}}{2} = 45^\circ$ \vspace{.3cm}
\end{center}

Ou seja, o angulo $\theta$ é igual a $45^\circ$ ou $\frac{\pi}{4}$ \vspace{.3cm}

Agora para descobrir a forma trigonométrica, vamos usar a formula: \vspace{.3cm}

\begin{center}
    $z=\rho(cos\theta + isen\theta)$ \vspace{.3cm}
    
    \text{Agora vamos substituir os valores na formula.} \vspace{.3cm}
    
    $z=2\sqrt{2}(cos\frac{\pi}{4} + isen\frac{\pi}{4})$ \vspace{.3cm}
    
    \text{Então esse é o resultado da forma trigonométrica da equação.} \vspace{.3cm}
\end{center}

Exemplo 2: \vspace{.3cm}

Dê a forma trigonométrica de $z=-1+i$

Vamos achar o valor do módulo: \vspace{.3cm}

\begin{center}
    $\rho=\sqrt{\sqrt{(-1)^2}+1^2} = \sqrt{2}$ \vspace{.3cm}
\end{center}

Agora vamos achar o argumento: \vspace{.3cm}

\begin{center}
    $sen\theta=\frac{1}{\sqrt{2}}$ \vspace{.cm}
\end{center}

Racionalizando, teremos: \vspace{.3cm}

\begin{center}
    $\frac{1}{\sqrt{2}} \cdot \frac{\sqrt{2}}{\sqrt{2}} = \frac{\sqrt{2}}{2}$ \vspace{.3cm}
\end{center}

Vamos achar o valor do cosseno: \vspace{.3cm}

\begin{center}
    $cos\theta=\frac{-1}{\sqrt{2}} = \frac{-\sqrt{2}}{2}$ \vspace{.3cm}
\end{center}

Vemos que o valor do cosseno ficou negativo, então com os devidos conhecimentos sobre o circulo trigonométrico, sabemos que o multiplo de $45^\circ$ é $135^\circ$. \vspace{.3cm}

Então o valor de $\theta$ vai ser $135^\circ$. \vspace{.3cm}

Agora vamos escrever na forma trigonométrica: \vspace{.3cm}

\begin{center}
    $z=\sqrt{2}(cos135^\circ + iseno135^\circ)$ \vspace{.3cm}
\end{center}

\subsection{Formas trigonométricas - Multiplicação e divisão}

Multiplicação \vspace{.3cm}

Para realizar a multiplicação, vamos usar a formula a seguir: \vspace{.3cm}

\begin{center}
    $z1 \cdot z2=\rho1\cdot\rho2(cos(\theta1+\theta2)+isen(\theta1+\theta2))$
\end{center}

Exemplo: \vspace{.3cm}

Seja $z=2(cos45^\circ +isen45^\circ)$ e $w=\sqrt{3}(cos30^\circ +isen30^\circ$, qual o resultado de $z\cdot w$ ? \vspace{.3cm}

Utilizando a formula que nos foi apresentada, chegaremos nesse resultado: \vspace{.3cm}

\begin{center}
    $z \cdot w = 2\sqrt{3}(cos75^\circ+isen75^\circ)$ \vspace{.3cm}
\end{center}

Divisão: \vspace{.3cm}

Para fazer a divisão, usaremos a formula a seguir: \vspace{.3cm}

\begin{center}
    $\frac{z1}{z2}=\frac{\rho1}{\rho2}(cos(\theta1-\theta2) + isen(\theta1 - \theta2))$ \vspace{.3cm}
\end{center}

Exemplo: \vspace{.3cm}

Seja $z=3(cos2\pi+isen2\pi)$ e $w=2(cos\frac{\pi}{3} + isen\frac{\pi}{3})$, calcule $\frac{z}{w}$

Utilizando a forma, teremos: \vspace{.3cm}

\begin{center}
    $\frac{z}{w}=\frac{3}{2}(cos\frac{5\pi}{3} + isen\frac{5\pi}{3})$ \vspace{.3cm}
\end{center}

\section{Potenciação - Formula de moivre}

Vamos falar sobre a $1^a$ formula de Moivre, que é sobre potenciação: \vspace{.3cm}

\begin{center}
    $z^2=\rho^n(cos(n \cdot \theta)+isen(n \cdot \theta))$ \vspace{.3cm}
\end{center}

A formula apresentando acima, é usada para calcular a potenciação.

Exemplo: \vspace{.3cm}

Seja $z=5(cos\frac{\pi}{3} + isen\frac{\pi}{3)}$, calcule $z^3$ \vspace{.3cm}

\begin{center}
    $z^3=5^3(cos(3\cdot \frac{\pi}{3}) + isen(3\cdot \frac{\pi}{3}))=$
    
    $z^3=125(cos\pi + isen\pi)$ \vspace{.3cm}
\end{center}

Exemplo 2: \vspace{.3cm}

Calcule $(\sqrt{3}+i)^1{}^0$ \vspace{.3cm}

Vamos encontrar o valor do módulo:

\begin{center}
    $\rho=\sqrt{\sqrt{(3)^2}+1^2} = 2$ \vspace{.3cm}
\end{center}

Agora vamos encontrar o valor do argumento: \vspace{.3cm}

\begin{center}
    $sen\theta=\frac{1}{2}=30^\circ$ \vspace{.3cm}
    
    $cos\theta=\frac{\sqrt{3}}{2}=30^\circ$ \vspace{.3cm}
\end{center}

Então o valor do argumento será: \vspace{.3cm}

\begin{center}
    $\theta=30^\circ$ ou $\frac{\pi}{6}$
\end{center}

Agora vamos determinar a forma trigonométrica: \vspace{.cm}

\begin{center}
    $z=2(cos \cdot \frac{\pi}{6} + isen \cdot \frac{\pi}{6})$
\end{center}

Agora vamos calcular $z^1{}^0$ \vspace{.3cm}

\begin{center}
    $z^1{}^0=2^1{}^0(cos(10 \cdot \frac{\pi}{6}) + isen(10 \cdot \frac{\pi}{6}))$ \vspace{.3cm}
    
    $z^1{}^0=1024(cos\frac{5\pi}{3}+isen\frac{5\pi}{3})$ \vspace{.3cm}
\end{center}


Com o conhecimento adquirido desse material, você estára apto a resolver problemas envolvendo números complexos.

\end{flushleft}
\end{large}
\end{document}
